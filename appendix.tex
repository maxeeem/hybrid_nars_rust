\chapter{Narsese Grammar} The grammar rules in this book are written in a variant of the Backus-Naur Form (BNF), specified as the following: \begin{itemize} \item Each rule has the format ``\(\langle symbol \rangle ::= expression\)'', where the $symbol$ is a \emph{nonterminal}, and the $expression$ consists of a sequence of symbols, as a substitution for the symbol. \item Symbols that never appear on a left side are \emph{terminals} that are specified by definitions in the book. \item Symbols without the ``\(\langle \rangle\)'' are used literally. Quotation makers are used to avoid confusion if a symbol is also used for other purposes. \item Expression ``\(exp_1 | exp_2\)'' indicates alternative substitutions. \item Expression ``[\(\langle symbol \rangle\)]'' indicates an optional symbol. \item Expression ``\(\langle symbol \rangle ^*\)'' indicates a symbol repeating zero or more times. \item Expression ``\(\langle symbol \rangle ^+\)'' indicates a symbol repeating one or more times. \end{itemize} The complete list of Narsese grammar rules is in Table \ref{Narsese-Grammar}. \begin{table}[htb] \tbl{The Complete Grammar of Narsese} {\(\begin{array}{rcl} \toprule \langle sentence \rangle & ::= & \langle judgment \rangle \; | \; \langle goal \rangle \; | \; \langle question \rangle \\ \langle judgment \rangle & ::= & [\langle tense \rangle] \langle statement \rangle . \, [\langle truth \mbox{-} value \rangle] \\ \langle goal \rangle & ::= & \langle statement \rangle ! \, [\langle desire \mbox{-} value \rangle] \\ \langle question \rangle & ::= & [\langle tense \rangle] \langle statement \rangle ? \, \; | \; \langle statement \rangle \mbox{?`} \\ \langle statement \rangle & ::= & (\langle term \rangle \, \langle copula \rangle \, \langle term \rangle ) \; | \; \langle term \rangle \\ && | \; (\neg \, \langle statement \rangle ) \\ && | \; (\wedge \, \langle statement \rangle \, \langle statement \rangle^+ ) \\ && | \; (\vee \, \langle statement \rangle \, \langle statement \rangle^+ ) \\ && | \; (\;,\, \langle statement \rangle \, \langle statement \rangle^+ ) \\ && | \; (\;;\, \langle statement \rangle \, \langle statement \rangle^+ ) \\ && | \; (\Uparrow\!\! \langle word \rangle \, \langle term \rangle^* ) \\ \langle copula \rangle & ::= & \rightarrow | \leftrightarrow | \Rightarrow | \Leftrightarrow \\ && | \; \instance \; | \; \property \; | \; \instanceProperty \\ && | \;\predictiveImplication \; | \; \retrospectiveImplication \; | \; \concurrentImplication \; | \;\predictiveEquivalance | \; \concurrentEquivalance \\ \langle tense \rangle & ::= & \predictiveImplication \; | \; \retrospectiveImplication \; | \; \concurrentImplication \\ \langle term \rangle & ::= & \langle word \rangle \; | \; \langle variable \rangle \; | \; \langle statement \rangle \\ && | \; \{ \langle term \rangle ^+ \} \; | \; [ \langle term \rangle ^+ ] \\ && | \; (\cap \, \langle term \rangle \, \langle term \rangle ^+ ) \\ && | \; (\cup \, \langle term \rangle \, \langle term \rangle ^+ ) \\ && | \; (- \, \langle term \rangle \, \langle term \rangle ) \\ && | \; (\ominus \, \langle term \rangle \, \langle term \rangle ) \\ && | \; (\times \, \langle term \rangle \, \langle term \rangle ^+ ) \\ && | \; (/ \, \langle term \rangle \, \langle term \rangle ^* \diamond \langle term \rangle ^* ) \\ && | \; (\backslash \, \langle term \rangle \, \langle term \rangle ^* \diamond \langle term \rangle ^* ) \\ \langle variable \rangle & ::= & \$ \!\langle word \rangle \; | \; \# [ \langle word \rangle ] \; | \; ? [ \langle word \rangle ] \\ \botrule \end{array}\)} \label{Narsese-Grammar} \end{table} Notes about the Narsese grammar: \begin{itemize} \item A $\langle word \rangle$ is a string of characters of a given alphabet. \item A $\langle truth \mbox{-} value \rangle$ and $\langle desire \mbox{-} value \rangle$ is a pair of numbers from $[0, 1] \times (0, 1)$. In communication between the system and its environment, it can be replaced by amounts of evidence or frequency interval. When omitted, default values are used. \item Most prefix operators in compound terms and compound statements can also be used in infix form. Operators can be moved to the front of the argument list. \end{itemize} The inference rules of NAL only recognize a subset of the grammar rules in Table \ref{Narsese-Grammar}, and the other sentences are either converted (e.g., those with a derived copula or a tense) or ignored (e.g., those with a variable term at an unsupported position). The symbols that appeared in Narsese grammar are listed in Table \ref{Narsese-Symbols}, with their ASCII code used in implementations. The grammar used in the previous implementations also has some variations (such as putting tense after statement). \begin{table}[htb] \tbl{The Symbols in Narsese Grammar} {\(\begin{array}{rcccc} \toprule \textbf{type\;} & \textbf{\;symbol\;} & \textbf{\;code\;} & \textbf{\;name\;} & \textbf{\;\;\;\;\;\;layer\;\;\;\;\;\;}\\ \colrule \mbox{\emph{basic copula}} & \rightarrow & \texttt{-->} & \mbox{inheritance} & \mbox{NAL-1}\\ & \leftrightarrow & \texttt{<->} & \mbox{similarity} & \mbox{NAL-2}\\ & \Rightarrow & \texttt{==>} & \mbox{implication} & \mbox{NAL-5}\\ & \Leftrightarrow & \texttt{<=>}& \mbox{equivalence} & \mbox{NAL-5}\\ \colrule \mbox{\emph{derived copula}} & \instance & \texttt{\{--} & \mbox{instance} & \mbox{NAL-2}\\ & \property & \texttt{--]} & \mbox{property} & \mbox{NAL-2}\\ & \instanceProperty & \texttt{\{-]} & \mbox{instance-property} & \mbox{NAL-2}\\ & \predictiveImplication & \texttt{=/>} & \mbox{predictive implication} & \mbox{NAL-7}\\ & \retrospectiveImplication & \texttt{=} \backslash \texttt{>} & \mbox{retrospective implication} & \mbox{NAL-7}\\ & \concurrentImplication & \texttt{=|>} & \mbox{concurrent implication} & \mbox{NAL-7}\\ & \predictiveEquivalance & \texttt{</>} & \mbox{predictive equivalence} & \mbox{NAL-7}\\ & \concurrentEquivalance & \texttt{<|>} & \mbox{concurrent equivalence} & \mbox{NAL-7}\\ \colrule \mbox{\emph{tense}} & \predictiveImplication & \texttt{:/:} & \mbox{future} & \mbox{NAL-7}\\ & \retrospectiveImplication & \texttt{:} \backslash \texttt{:} & \mbox{past} & \mbox{NAL-7}\\ & \concurrentImplication & \texttt{:|:} & \mbox{present} & \mbox{NAL-7}\\ \colrule \mbox{\emph{term connector}} & \{ \} & \texttt{\{\}} & \mbox{extensional set} & \mbox{NAL-2}\\ & \mbox{[ ]} & \texttt{[]} & \mbox{intensional set} & \mbox{NAL-2}\\ & \cap & \texttt{\&} & \mbox{extensional intersection} & \mbox{NAL-3}\\ & \cup & \texttt{|} & \mbox{intensional intersection} & \mbox{NAL-3}\\ & - & \texttt{-} & \mbox{extensional difference} & \mbox{NAL-3}\\ & \ominus & \texttt{\textasciitilde} & \mbox{intensional difference} & \mbox{NAL-3}\\ & \times & \texttt{*} & \mbox{product} & \mbox{NAL-4}\\ & / & \texttt{/} & \mbox{extensional image} & \mbox{NAL-4}\\ & \backslash & \backslash & \mbox{intensional image} & \mbox{NAL-4}\\ & \diamond & \texttt{\_} & \mbox{image place-holder} & \mbox{NAL-4}\\ \colrule \mbox{\emph{statement connector}} & \neg & \texttt{--} & \mbox{negation} & \mbox{NAL-5} \\ & \wedge & \texttt{\&\&} & \mbox{conjunction} & \mbox{NAL-5} \\ & \vee & \texttt{||} & \mbox{disjunction} & \mbox{NAL-5} \\ & , & \texttt{,} & \mbox{sequential conjunction} & \mbox{NAL-7}\\ & ; & \texttt{;} & \mbox{parallel conjunction} & \mbox{NAL-7}\\ \colrule \mbox{\emph{term prefix}} & \$ & \texttt{\$} & \mbox{independent variable} & \mbox{NAL-6}\\ & \# & \texttt{\#} & \mbox{dependent variable} & \mbox{NAL-6}\\ & ? & \texttt{?} & \mbox{query variable} & \mbox{NAL-6}\\ & \Uparrow & \texttt{\char94} & \mbox{operator} & \mbox{NAL-8}\\ \colrule \mbox{\emph{punctuation}} & . & \texttt{.} & \mbox{judgment} & \mbox{NAL-8}\\ & ! & \texttt{!} & \mbox{goal} & \mbox{NAL-8}\\ & ? & \texttt{?} & \mbox{question on truth-value} & \mbox{NAL-8}\\ & \mbox{?`} & \texttt{@} & \mbox{question on desire-value} & \mbox{NAL-8}\\ \botrule \end{array}\)} \label{Narsese-Symbols} \end{table} \chapter{NAL Inference Rules} The inference rules of NAL are classified into several categories according to their syntactic features. \textbf{(1) Local inference rules:} Each of these rules directly processes an inference task according to the available information stored locally in the concept representing the content of the task. These rules are usually applied before the other rules are attempted on the task. \begin{description} \item[(1.1) Revision.] When two judgments have disjoint evidential bases, the \emph{revision rule} is applied to produce a new judgment with the same statement and a truth-value calculated by $F_{rev}$. The same revision process is applied to two desire-values of the same statement. \item[(1.2) Choice.] When the task is a \emph{question} (or a \emph{goal}), the system matches it with the existing judgments on the same statement to find candidate solutions. If two candidates contain the same statement, the one with the higher \emph{confidence} is chosen; if two candidates suggest different instantiations to the variable(s) in the task, the one with high \emph{expectation} $e$ and low \emph{complexity} $n$ is chosen, using the ranking formula $e/n^r$ (with $r=1$ as the default). \item[(1.3) Decision.] A candidate goal is turned into an active goal when the expectation of its desire-value is higher than a personality parameter $d$ (\(d > 0.5\)). \end{description} \newpage \textbf{(2) Two-premise inference rules:} Each of these rules takes two judgments $J_1$ and $J_2$ as premises, and derives a judgment $J$ as a conclusion, with a truth-value calculated by a function $F$. $F'$ is $F$ with the order of the two premises switched. The same premises may trigger multiple rules. \begin{description} \item[(2.1) First-order syllogism,] in Table \ref{First-Order-Syllogism}, are defined on copulas \emph{inheritance} and \emph{similarity}. \begin{table}[htb] \tbl{The First-Order Syllogistic Rules} {\(\begin{array}{cc|cccccc} \toprule J_2 \;\;\;\; \backslash \;\;\;\; J_1 &\;&\;& M \rightarrow P \; \langle f_1, \, c_1 \rangle &\;& P \rightarrow M \; \langle f_1, \, c_1 \rangle &\;& M \leftrightarrow P \; \langle f_1, \, c_1 \rangle \\ \colrule &&& S \rightarrow P \; \langle F_{ded} \rangle && S \rightarrow P \; \langle F_{abd}\rangle && S \rightarrow P \; \langle F'_{ana}\rangle \\ S \rightarrow M \; \langle f_2, \, c_2 \rangle &&& P \rightarrow S \; \langle F'_{exe}\rangle && P \rightarrow S \; \langle F'_{abd}\rangle && \\ &&& && S \leftrightarrow P \; \langle F'_{com}\rangle && \\ \colrule &&& S \rightarrow P \; \langle F_{ind}\rangle && S \rightarrow P \; \langle F_{exe}\rangle && \\ M \rightarrow S \; \langle f_2, \, c_2\rangle &&& P \rightarrow S \; \langle F'_{ind}\rangle && P \rightarrow S \; \langle F'_{ded}\rangle && P \rightarrow S \; \langle F'_{ana}\rangle \\ &&& S \leftrightarrow P \; \langle F_{com}\rangle && && \\ \colrule &&& S \rightarrow P \; \langle F_{ana}\rangle && && \\ S \leftrightarrow M \; \langle f_2, \, c_2\rangle &&& && P \rightarrow S \; \langle F_{ana} \rangle && \\ &&& && && S \leftrightarrow P \; \langle F_{res} \rangle \\ \botrule \end{array}\)} \label{First-Order-Syllogism} \end{table} \item[(2.2) Higher-order syllogism] can be obtained by replacing the copulas \emph{inheritance} and \emph{similarity} in Table \ref{First-Order-Syllogism} with \emph{implication} and \emph{equivalence}, respectively. \item[(2.3) Conditional syllogism,] in Table \ref{Conditional-Syllogism}, are based on the nature of conditional statements. \begin{table}[htb] \tbl{The Conditional Syllogistic Rules} {\(\begin{array}{rcrcrcl} \toprule J_1 \; \langle f_1, \, c_1 \rangle &\;& J_2 \; \langle f_2, \, c_2 \rangle &\;\;& J &\;& F \\ \colrule S && S \Leftrightarrow P && P && F_{ana} \\ S && P && S \Leftrightarrow P && F_{com} \\ S \Rightarrow P && S && P && F_{ded} \\ P \Rightarrow S && S && P && F_{abd} \\ P && S && S \Rightarrow P && F_{ind} \\ (C \wedge S) \Rightarrow P && S && C \Rightarrow P && F_{ded} \\ (C \wedge S) \Rightarrow P && C \Rightarrow P && S && F_{abd} \\ C \Rightarrow P && S && (C \wedge S) \Rightarrow P && F_{ind} \\ (C \wedge S) \Rightarrow P && M \Rightarrow S && (C \wedge M) \Rightarrow P && F_{ded} \\ (C \wedge S) \Rightarrow P && (C \wedge M) \Rightarrow P && M \Rightarrow S && F_{abd} \\ (C \wedge M) \Rightarrow P && M \Rightarrow S && (C \wedge S) \Rightarrow P && F_{ind} \\ \botrule \end{array}\)} \label{Conditional-Syllogism} \end{table} \newpage \item[(2.4) Composition rules,] in Table \ref{Composition-Rules}, introduce new compounds in the conclusion. \begin{table}[htb] \tbl{The Composition Rules} {\(\begin{array}{rcrcrcl} \toprule J_1 \; \langle f_1, \, c_1 \rangle &\;& J_2 \; \langle f_2, \, c_2 \rangle &\;\;& J &\;& F \\ \colrule M \rightarrow T_1 && M \rightarrow T_2 && M \rightarrow (T_1 \cap T_2) && F_{int} \\ && && M \rightarrow (T_1 \cup T_2) && F_{uni} \\ && && M \rightarrow (T_1 - T_2) && F_{dif} \\ && && M \rightarrow (T_2 - T_1) && F'_{dif} \\ \colrule T_1 \rightarrow M && T_2 \rightarrow M && (T_1 \cup T_2) \rightarrow M && F_{int} \\ && && (T_1 \cap T_2) \rightarrow M && F_{uni} \\ && && (T_1 \ominus T_2) \rightarrow M && F_{dif} \\ && && (T_2 \ominus T_1) \rightarrow M && F'_{dif} \\ \colrule M \Rightarrow T_1 && M \Rightarrow T_2 && M \Rightarrow (T_1 \wedge T_2) && F_{int} \\ && && M \Rightarrow (T_1 \vee T_2) && F_{uni} \\ \colrule T_1 \Rightarrow M && T_2 \Rightarrow M && (T_1 \vee T_2) \Rightarrow M && F_{int} \\ && && (T_1 \wedge T_2) \Rightarrow M && F_{uni} \\ \colrule T_1 && T_2 && T_1 \wedge T_2 && F_{int} \\ && && T_1 \vee T_2 && F_{uni} \\ \botrule \end{array}\)} \label{Composition-Rules} \end{table} \item[(2.5) Decomposition rules] are the opposite of the composition rules. Each decomposition rule comes from a high-level theorem of the form \((S_1 \wedge S_2) \Longrightarrow S\) (in Table \ref{Decomposition}) where $S_1$ is a statement about a compound, $S_2$ is a statement about a component of the compound, while $S$ is the statement about the other component. \begin{table}[htb] \tbl{The Decomposition Rules} {\(\begin{array}{rcrcr} \toprule S_1 &\;& S_2 &\;& S \\ \colrule \neg (M \rightarrow (T_1 \cap T_2)) && M \rightarrow T_1 && \neg (M \rightarrow T_2) \\ M \rightarrow (T_1 \cup T_2) && \neg (M \rightarrow T_1) && M \rightarrow T_2 \\ \neg (M \rightarrow (T_1 - T_2)) && M \rightarrow T_1 && M \rightarrow T_2 \\ \neg (M \rightarrow (T_2 - T_1)) && \neg (M \rightarrow T_1) && \neg (M \rightarrow T_2) \\ \neg ((T_1 \cup T_2) \rightarrow M) && T_1 \rightarrow M && \neg (T_2 \rightarrow M) \\ (T_1 \cap T_2) \rightarrow M && \neg(T_1 \rightarrow M) && T_2 \rightarrow M \\ \neg ((T_1 \ominus T_2) \rightarrow M) && T_1 \rightarrow M && T_2 \rightarrow M \\ \neg ((T_2 \ominus T_1) \rightarrow M) && \neg (T_1 \rightarrow M) && \neg (T_2 \rightarrow M) \\ \neg (T_1 \wedge T_2) && T_1 && \neg T_2 \\ T_1 \vee T_2 && \neg T_1 && T_2 \\ \botrule \end{array}\)} \label{Decomposition} \end{table} \end{description} \newpage \textbf{(3) One-premise inference rules:} Each of these rules carries out inference from a judgment $J_1$ as premise to a judgment $J$ as conclusion, with a truth-value calculated by function $F$. \begin{description} \item[(3.1) Immediate inference,] Table \ref{Immediate-Inference} contains the inference rules each of which only takes one premise. \begin{table}[htb] \tbl{The Immediate Inference Rules} {\(\begin{array}{rcrcl} \toprule J_1 &\;& J &\;& F \\ \colrule S && \neg S && F_{neg} \\ S \rightarrow P && P \rightarrow S && F_{cnv} \\ S \Rightarrow P && P \Rightarrow S && F_{cnv} \\ \;\;\;\;\;\;\;\;\;\; S \Rightarrow P && (\neg P) \Rightarrow (\neg S) && F_{cnt} \;\;\;\;\;\;\;\;\;\;\\ \botrule \end{array}\)} \label{Immediate-Inference} \end{table} \item[(3.2) Structural inference] is carried out according to the literal meaning of compound terms. Each inference rule of this type takes a definition or theorem in IL (summarized in Table \ref{Inheritance-Theorems}, \ref{Similarity-Theorems}, \ref{Implication-Theorems}, and \ref{Equivalence-Theorems} for each of the four copulas) as a Narsese judgment $J_2$ with truth value \(\langle 1, 1 \rangle \). Using this analytical truth and an empirical judgment $J_1$ as premises, a two-premise inference rule can derive a conclusion $J$. Since $J_2$ is not explicitly represented, this rule effectively derives $J$ from a single premise $J_1$. \begin{table}[htb] \tbl{The Inheritance Theorems} {\(\begin{array}{rcl} \toprule term_1 & \rightarrow & term_2 \\ \colrule (T_1 \cap T_2) && T_1 \\ T_1 && (T_1 \cup T_2) \\ (T_1 - T_2) && T_1 \\ T_1 && (T_1 \ominus T_2) \\ \;\;\;\;\;\;\;\;\;\; ((R \, / \, T) \times T) && R \\ R && ((R \, \backslash \, T) \times T) \;\;\;\;\;\;\;\;\;\; \\ \botrule \end{array}\)} \label{Inheritance-Theorems} \end{table} \end{description} \begin{table}[htb] \tbl{The Similarity Theorems} {\(\begin{array}{rcl} \toprule term_1 & \leftrightarrow & term_2 \\ \colrule \neg (\neg T) && T \\ (\cup \; \{T_1\} \; \cdots \; \{T_n\}) && \{T_1, \; \cdots , \; T_n\} \\ (\cap \; [T_1] \; \cdots \; [T_n]) && [T_1, \; \cdots , \; T_n] \\ (\{T_1, \; \cdots , \; T_n\} - \{T_n\}) && \{T_1, \; \cdots , \; T_{n-1}\} \\ ([T_1, \; \cdots , \; T_n] \ominus [T_n]) && [T_1, \; \cdots , \; T_{n-1}] \\ ((T_1 \times T_2) \, / \, T_2) && T_1 \\ ((T_1 \times T_2) \, \backslash \, T_2) && T_1 \\ \botrule \end{array}\)} \label{Similarity-Theorems} \end{table} \begin{table}[htb] \tbl{The Implication Theorems} {\(\begin{array}{rcl} \toprule statement_1 & \Rightarrow & statement_2 \\ \colrule S \leftrightarrow P && S \rightarrow P\\ S \Leftrightarrow P && S \Rightarrow P\\ S_1 \wedge S_2 && S_1 \\ S_1 && S_1 \vee S_2 \\ S \rightarrow P && (S \cup M) \rightarrow (P \cup M) \\ S \rightarrow P && (S \cap M) \rightarrow (P \cap M) \\ S \leftrightarrow P && (S \cup M) \leftrightarrow (P \cup M) \\ S \leftrightarrow P && (S \cap M) \leftrightarrow (P \cap M) \\ S \Rightarrow P && (S \vee M) \Rightarrow (P \vee M) \\ S \Rightarrow P && (S \wedge M) \Rightarrow (P \wedge M) \\ S \Leftrightarrow P && (S \vee M) \Leftrightarrow (P \vee M) \\ S \Leftrightarrow P && (S \wedge M) \Leftrightarrow (P \wedge M) \\ S \rightarrow P && (S - M) \rightarrow (P - M) \\ S \rightarrow P && (M - P) \rightarrow (M - S) \\ S \rightarrow P && (S \ominus M) \rightarrow (P \ominus M) \\ S \rightarrow P && (M \ominus P) \rightarrow (M \ominus S) \\ S \leftrightarrow P && (S - M) \leftrightarrow (P - M) \\ S \leftrightarrow P && (M - P) \leftrightarrow (M - S) \\ S \leftrightarrow P && (S \ominus M) \leftrightarrow (P \ominus M) \\ S \leftrightarrow P && (M \ominus P) \leftrightarrow (M \ominus S) \\ M \rightarrow (T_1 - T_2) && \neg(M \rightarrow T_2) \\ (T_1 \ominus T_2) \rightarrow M && \neg(T_2 \rightarrow M) \\ S \rightarrow P && (S \; / \; M) \rightarrow (P \; / \; M) \\ S \rightarrow P && (S \; \backslash \; M) \rightarrow (P \; \backslash \; M) \\ S \rightarrow P && (M \; / \; P) \rightarrow (M \; / \; S) \\ S \rightarrow P && (M \; \backslash \; P) \rightarrow (M \; \backslash \; S) \\ \botrule \end{array}\)} \label{Implication-Theorems} \end{table} \begin{table}[htb] \tbl{The Equivalence Theorems} {\(\begin{array}{rcl} \toprule statement_1 & \Leftrightarrow & statement_2 \\ \colrule S \leftrightarrow P && (S \rightarrow P) \wedge (P \rightarrow S) \\ S \Leftrightarrow P && (S \Rightarrow P) \wedge (P \Rightarrow S) \\ S \leftrightarrow P && \{S\} \leftrightarrow \{P\} \\ S \leftrightarrow P && [S] \leftrightarrow [P] \\ S \rightarrow \{P\} && S \leftrightarrow \{P\} \\ \mbox{[}S\mbox{]} \rightarrow P && [S] \leftrightarrow P \\ (S_1 \times S_2) \rightarrow (P_1 \times P_2) && (S_1 \rightarrow P_1) \wedge (S_2 \rightarrow P_2) \\ (S_1 \times S_2) \leftrightarrow (P_1 \times P_2) && (S_1 \leftrightarrow P_1) \wedge (S_2 \leftrightarrow P_2) \\ S \rightarrow P && (M \times S) \rightarrow (M \times P) \\ S \rightarrow P && (S \times M) \rightarrow (P \times M) \\ S \leftrightarrow P && (M \times S) \leftrightarrow (M \times P) \\ S \leftrightarrow P && (S \times M) \leftrightarrow (P \times M) \\ (\times \; T_1 \; T_2) \rightarrow R && T_1 \rightarrow (/ \; R \; \diamond \; T_2) \\ (\times \; T_1 \; T_2) \rightarrow R && T_2 \rightarrow (/ \; R \; T_1 \; \diamond) \\ R \rightarrow (\times \; T_1 \; T_2) && (\backslash \; R \; \diamond \; T_2) \rightarrow T_1 \\ R \rightarrow (\times \; T_1 \; T_2) && (\backslash \; R \; T_1 \; \diamond) \rightarrow T_2 \\ S_1 \Rightarrow (S_2 \Rightarrow S_3) && (S_1 \wedge S_2) \Rightarrow S_3 \\ \neg(S_1 \wedge S_2) && (\neg S_1) \vee (\neg S_2) \\ \neg(S_1 \vee S_2) && (\neg S_1) \wedge (\neg S_2) \\ S_1 \Leftrightarrow S_2 && (\neg S_1) \Leftrightarrow (\neg S_2) \\ \botrule \end{array}\)} \label{Equivalence-Theorems} \end{table} \textbf{(4) Meta-level rules:} Each of these rules specifies how to use the other rules defined above for additional functions. \begin{description} \item[(4.1) Question derivation.] A question $Q$ and a judgment $J$ produce a derived question $Q'$, if and only if the answer to $Q'$, call it $J'$, can be used with $J$ to derive an answer to $Q$ by a two-premise inference rule; a question $Q$ by itself produces a derived question $Q'$, if and only if the answer to $Q'$, call it $J'$, can be used to derive an answer to $Q$ by a one-premise inference rule. \item[(4.2) Goal derivation.] A goal $G$ and a judgment $J$ produce a derived goal $G'$, if and only if the solution to $G'$, call it $J'$, can be used with $J$ to derive a solution to $G$ by a two-premise inference rule; a goal $G$ by itself produces a derived goal $G'$, if and only if the solution to $G'$, call it $J'$, can be used to derive a solution to $G$ by a one-premise inference rule. In both cases, the desire-value of $G'$ is derived as the truth-value of \(G' \Rightarrow D\) from the desire-value of $G$, as the truth-value of \(G \Rightarrow D\), as well as the truth-value of $J$ (if it is involved). A derived goal is initially treated as a candidate and needs to go through decision-making to become an active goal. \item[(4.3) Variable substitution.] All occurrences of an independent variable term in a statement can be substituted by another term (constant or variable); all occurrences of a term (constant or variable) in a statement can be substituted by a dependent variable term. The reverse cases of these substitutions are limited to the cases discussed in NAL-6. A query variable in a question can be substituted by a constant term in a judgment. \item[(4.4) Temporal inference.] Temporal inference is carried out by processing the logical factor and the temporal factor in the premises in parallel. First, temporal variants of IL rules are obtained by turning some statements in the premises into events by adding temporal order among them, and the conclusion must keep the same temporal information. Then these rules are extended into strong NAL rules by using the same truth-value function as in the lower layers. The rules of weak inference are formed as the reverse of the strong rules in the lower layers. When the belief has an occurrence time that is different from the task, its truth-value is either eternalized or projected before being used as a premise. \end{description} \chapter{NAL Truth-Value Functions} The relations among the three forms of uncertainty measurements are summarized in Table \ref{Uncertainty-2}, which can be extended to include \(w^- = w - w^+\) and \(i = u - l\). \begin{table}[htb] \tbl{The Relations Among Uncertainty Measurements} {\(\begin{array}{cc|clclcl} \toprule \mbox{to} \;\; \backslash \;\; \mbox{from} & \; & \; & \; \{w^+, \, w\} & \; & \; \langle f, \, c \rangle & \; & \; [ \, l, \, u \, ] \; (\mbox{and} \; i) \\ \colrule \{w^+, \, w\} &&& && w^+ = k \times f \times c \,/\,(1-c) && w^+ = k \times l\,/\,i \\ &&& && w = k \times c \,/\,(1-c) && w = k \times (1-i)\,/\,i \\ &&&&&& \\ \langle f, \, c \rangle &&& f = w^+ \,/\, w && && f = l \,/\, (1-i) \\ &&& c = w \,/\, (w+k) && && c = 1-i \\ &&&&&& \\ \mbox{[}l, \, u\mbox{]} &&& l = w^+ \,/\, (w+k) && l = f \times c && \\ &&& u = (w^++k) \,/\, (w+k) && u = 1 - c \times(1-f) && \\ \botrule \end{array}\)} \label{Uncertainty-2} \end{table} For independent extended Boolean variables in [0, 1], the extended Boolean operators are defined in Table \ref{Uncertainty-3}. \begin{table}[htb] \tbl{The Extended Boolean Operators} {\(\begin{array}{rcl} \toprule not(x) & = & 1 - x \\ and(x_1, \cdots, x_n) & = & x_1 \times \cdots \times x_n \\ or(x_1, \cdots, x_n) & = & 1 - (1-x_1) \times \cdots \times (1-x_n) \\ \botrule \end{array}\)} \label{Uncertainty-3} \end{table} All truth-value functions in inference rules are summarized in Table \ref{NAL-Functions}, in their simplest form, so different types of uncertainty measurements are mixed. The functions are classified according to the type of inference. \begin{table}[b] \tbl{The Truth-Value Functions of Inference Rules} {\(\begin{array}{llllll} \toprule \textbf{type} & \textbf{inference} & \textbf{name} \;\;\;\; &&& \textbf{function} \\ \colrule \mbox{\emph{local inference}} & \mbox{revision} & F_{rev} & w^+ & = & w^+_1 + w^+_2 \\ && & w^- & = & w^-_1 + w^-_2 \\ & \mbox{expectation} & F_{exp} & e & = & c(f-0.5) + 0.5 \\ \colrule \mbox{\emph{immediate inference} \;\;} & \mbox{negation} & F_{neg} & w^+ & = & w^-_1 \\ && & w^- & = & w^+_1 \\ & \mbox{conversion} & F_{cnv} & w^+ & = & and(f_1, c_1) \\ && & w^- & = & 0 \\ & \mbox{contraposition} & F_{cnt} & w^+ & = & 0 \\ && & w^- & = & and((not(f_1), c_1) \\ & \mbox{eternalization} & F_{etn} & d & = & 1/(c+k) \\ & \mbox{projection} & F_{prj} & d & = & 2s / (2s + v) \\ \colrule \mbox{\emph{strong syllogism}} & \mbox{deduction} & F_{ded} & f & = & and(f_1, f_2) \\ && & c & = & and(f_1, f_2, c_1, c_2) \\ & \mbox{analogy} & F_{ana} & f & = & and(f_1, f_2) \\ && & c & = & and(f_2, c_1, c_2) \\ & \mbox{resemblance} & F_{res} & f & = & and(f_1, f_2) \\ && & c & = & and(or(f_1, f_2), c_1, c_2) \\ \colrule \mbox{\emph{weak syllogism}} & \mbox{abduction} & F_{abd} & w^+ & = & and(f_1, f_2, c_1, c_2) \\ && & w & = & and(f_1, c_1, c_2) \\ & \mbox{induction} & F_{ind} & w^+ & = & and(f_1, f_2, c_1, c_2) \\ && & w & = & and(f_2, c_1, c_2) \\ & \mbox{exemplification} & F_{exe} & w^+ & = & and(f_1, f_2, c_1, c_2) \\ && & w & = & and(f_1, f_2, c_1, c_2) \\ & \mbox{comparison} & F_{com} & w^+ & = & and(f_1, f_2, c_1, c_2) \\ && & w & = & and(or(f_1, f_2), c_1, c_2) \\ \colrule \mbox{\emph{term composition}} & \mbox{intersection} & F_{int} & f & = & and(f_1, f_2) \\ && & c & = & and(c_1, c_2) \\ & \mbox{union} & F_{uni} & f & = & or(f_1, f_2) \\ && & c & = & and(c_1, c_2) \\ & \mbox{difference} & F_{dif} & f & = & and(f_1, not(f_2)) \\ && & c & = & and(c_1, c_2) \\ \botrule \end{array}\)} \label{NAL-Functions} \end{table} \chapter{Mental Operators} Table \ref{mental-operations} contains the mental operations\index{NAL-9!mental operations} discussed in Chapter 13. Since the related research is still ongoing, this list is by no means final. \begin{table}[htb] \tbl{Mental Operations} {\begin{tabular}{lll} \toprule \textbf{Operator} & \textbf{Argument} & \textbf{Function} \\ \colrule believed & statement & find the truth-value of a statement \\ desired & statement & find the desire-value of a statement \\ recalled & term & find the priority of the concept identified by the term \\ satisfied && feel the average achieving level of the active goals \\ novel && feel the average novelty level of the recent inputs \\ busy && feel the average priority level of the recent tasks \\ healthy && feel the average achieving level of the recent operations \\ happy && feel the overall appraisal level of the system \\ doubt & statement & decrease the confidence in the truth-value of a statement \\ hesitate & statement & decrease the confidence in the desire-value of a statement \\ assume & statement & temporarily accept a statement as a belief \\ believe & statement & create a judgment with the statement \\ desire & statement & create a goal with the statement \\ ask & statement & create a question on the truth-value of the statement \\ inquire & statement & create a question on the desire-value of the statement \\ activate & term & increase the priority of the concept identified by the term \\ think && run a working cycle of the memory \\ observe & channel & run a working cycle of the channel \\ \botrule \end{tabular}} \label{mental-operations} \end{table}